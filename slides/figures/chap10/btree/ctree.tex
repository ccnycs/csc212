% Red-black tree
% Author: Madit
%%\documentclass{article}
\documentclass[preview]{standalone}
\usepackage{tikz}
%%%<
\usepackage{verbatim}
\usepackage[active,tightpage]{preview}
\usepackage{tikz-qtree}

\PreviewEnvironment{tikzpicture}
\setlength{\PreviewBorder}{10pt}
\begin{comment}
Adapated from http://www.texample.net/tikz/examples/red-black-tree/
\end{comment}

\usetikzlibrary{arrows}

\definecolor{rpurple}{RGB}{200,191,231}
\definecolor{mblue}{RGB}{153,217,234}
\definecolor{lyellow}{RGB}{239,228,176}

\tikzset{
  treenode/.style = {align=center, text centered,
    font=\sffamily,  minimum size=2em},
  root/.style = {treenode, circle, black, draw=black, fill=rpurple},
  middle/.style = {treenode, circle, black, draw=black, fill=mblue},
  leaf/.style = {treenode, circle, black, draw=black, fill=lyellow}
}

\begin{document}
\begin{tikzpicture}[grow=down]
%\tikzset{level 1/.style={level distance=40pt, sibling distance=60}}
%\tikzset{level 2/.style={level distance=40pt, sibling distance=50}}
%\tikzset{level 2/.style={level distance=40pt, sibling distance=60}}

{\Tree [.\node[root]{15}; 
           [.\node[middle]{17};
                [.\node[middle]{9};
                  [.\node[leaf]{13};]
                  [.\node[leaf]{33};]
                ]
                [.\node[middle]{53};
                 [.\node[leaf]{16};]
                 \edge[style={draw=none}]; \node[style={draw=none}]{};
                ]
           ]
           [.\node[middle]{11};
                [.\node[leaf]{4};]
                [.\node[leaf]{41};]
           ]
        ] 
]}

\end{tikzpicture}
\end{document}
